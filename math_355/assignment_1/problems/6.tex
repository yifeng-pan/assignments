\section{}
    \subsection{Find an explict injection $f: \Q \to \Z$}
        Let $q \in \Q$.
        Let $a \in \Z, b \in \N$ such that $(a,b) \in \Q$ is the irreducible fraction of $q$.
            \footnote{I've been instructed to clarify the definition of "irreducible fraction", 
            I've provided an attempt on page~\pageref{proof:IF}.}
        Let $m = \max(\abs{a},b)$.
        Let $n = 10$.
        Let $d = \floor{\log_{n}(m)} + 1$ 
        (where $\floor{x}$ means floor of $x$).
        \[
            f(q) = 
            \begin{cases}
                n^{2d} + n^d a + b & \text{ if $a \geq 0$} \\ 
                -(n^{2d} + n^d \abs{a} + b) & \text{ if $a < 0$}
            \end{cases}
        \]
        It's easy to see that $f(q) \in\Z$, 
        as $n, a, b, d \in \Z, d \geq 1$.
        
        Let $f^{-1}: \{f(x) | x \in \Q \} \to \Q$.
        Let $z \in \set{f(x) | x \in \Q}$.
        Let $n = 10$.
        Let $d' = \floor{\log_n \abs{z}} / 2$.
        Let $z' = \abs{z} \mod n^{2d'}$.
        $$
            \text{Let } a' = 
            \begin{cases}
                \floor{z' n^{-d'}} &\text{ if $z \geq 0$} \\ 
                -\floor{z' n^{-d'}} &\text{ if $z < 0$}
            \end{cases}
        $$
        Let $b' = z' \mod n^{d'}$.
        \[
            f^{-1}(z) = (a',b')
        \]
        Where $(a',b') \in \Q$.
        Because there is a function from the image of $f$ to the domain of $f$,
        $f$ is injective.

        Proof that $f^{-1}(f(q)) = q, \forall q \in \Q$: \\ 
        The following is for the non-negative case, the negative case can be proven similarly. \\ 
        $f^{-1}(f(q)) = f^{-1}(n^{2d} + n^d a + b)$.
        In this case $d' = \floor{\log_n \abs{n^{2d} + n^d a + b}} / 2 = 2d / 2 = d$.
        \footnote{$a < n^d, b < n^d$ by construction, as $d = \floor{\log_n(\max(\abs{a}, b))} + 1$.}
        And $z' = (n^{2d} + n^d a + b) \mod n^{2d} = n^d a + b$.
        Then $a' = \floor{(n^d a + b) n^{-d'}} = \floor{a + b n^{-d'}} = a$.
        And $b' = (n^d a + b) \mod n^{d'} = b$.
        So $f^{-1}(f(q)) = f^{-1}(n^{2d} + n^d a + b) = (a', b') = (a, b)$,
        which by construction of $a,b$ is in the same equivelence class as $q$.

        In fact, $n$ can be any integer $\geq 2$.
        \begin{multicols}{2}
            For example: For $n = 10$: \\ 
                $f(-137/100003) = -1000137100003$. \\ 
                $f^{-1}(-1000137100003) = -137/100003$.
    
            For $n = 2$: \\ 
                $f(-137/100003) = -17197926051$. \\ 
                $f^{-1}(-17197926051) = -137/100003$.
    
            % For $n = 100003$: \\ 
            %     $f(137/100003) = 100012001910093101317$ \\ 
            %     $f^{-1}(100012001910093101317) = 137/100003$.
    
            % For $n = 56748$: \\
            %     $f(-6257 / 241) = -3575407981$. \\ 
            %     $f^{-1}(-3575407981) = -6257 / 241$.
        \end{multicols}

    \newpage
    \subsection{Find an explict injection $g: \Q \times \Q \to \Z$}
        Let $q_1, q_2 \in \Q$.
        Let $a_1, a_2 \in \Z$ and $b_1, b_2 \in \N$ such that 
        $(a_1, b_1), (a_2,b_2) \in \Q$ are the irreducible fractions of 
        $q_1$ and $q_2$ respectively.
        Let $a_1' = \abs{a_1}$.
        Let $a_2' = \abs{a_2}$.
        For $i \in \set{1,2}$, let
        \[
            s_i = 
            \begin{cases}
                0 & \text{ if $a_i \geq 0$} \\ 
                1 & \text{ if $a_i < 0$} \\ 
            \end{cases}
        \]
        Let 
        \[
            g(q_1, q_2) =
                (2^{a_1'}) (3^{a_2'}) (5^{b_1}) (7^{b_2}) (11^{s_1}) (13^{s_2})
        \]
        It's easy to see that $g(q) \in\Z$.

        Proof that $g$ is injective: \\ 
        Suppose $g(q_1,q_2) = g(\bar q_1, \bar q_2)$.
        This means: 
        \begin{align*}
            &(2^{a_1'}) (3^{a_2'}) (5^{b_1}) (7^{b_2}) (11^{s_1}) (13^{s_2})\\
            = &(2^{\bar a_1'}) (3^{\bar a_2'}) (5^{\bar b_1}) (7^{\bar b_2}) (11^{\bar s_1}) (13^{\bar s_2})
        \end{align*}
        Due to the unique-prime-factorization theorem, $a_1' = \bar a_1', a_2' = \bar a_2'$ and so on.
        From the construction of $a_i'$ and $s_i$ in $g$ we know: 
        \[
            a_i = 
            \begin{cases}
                a_i' & \text{ if $s_i = 0$} \\ 
                -a_i' & \text{ if $s_i = 1$} \\ 
            \end{cases}
        \]
        Because $a_1' = \bar a_1'$ and $s_1 = \bar s_1$,
        so $a_1 = \bar a_1$.
        And similarly $a_2 = \bar a_2$.
        Therefore: 
        \[
            [q_1,q_2]
            = [(a_1, b_1), (a_2, b_2)]
            = [(\bar a_1, \bar b_1), (\bar a_2, \bar b_2)]
            = [\bar q_1, \bar q_2]
        \]

        
        % Let $g^{-1}: \{g(x) | x \in \Q \} \to \Q$. \\ 
        % Let $z,\in \Z, a_1, a_2, b_1, b_2 \in \N, s_1, s_2 \in \set{0,1}$, such that
        % $z = (2^{a_1}) (3^{a_2}) (5^{b_1}) (7^{b_2}) (11^{s_1}) (13^{s_2})$. 
        % We can say this, as the image of $g$ 
        % can only contain numbers with the first $6$ primes as prime factors,
        % And the domain for the exponents matches accordingly to the construction of $g$ as well.
        % Now for $i \in \set{0,1}$, let 
        % \[
        %     a_i' = 
        %     \begin{cases}
        %         a_i & \text{ if $s_i = 0$} \\ 
        %         -a_i & \text{ if $s_i = 1$} \\ 
        %     \end{cases}
        % \]
        % Let
        % \[
        %     g^{-1}(z) = \vectorvalue{(a_1', b_1), (a_2', b_2)}
        % \]
    
    % \newpage
    \subsection{Find an explict injection $h: \Z \times \Z \times \Z \to \Z \times \Z$}
        The idea of this function is similar to (6a): 
        "Allocate memory" with a leading digit, then concatenate some number using some base ($n$).
        It this case, an additional digit is allocated per number, to indicate the signs.
        And the third input is used to generate $n$ (the key),
        and the output is the coded message and the key. 

        Let $z_1, z_2, z_3 \in \Z$.
        For $j \in \set{1,2,3}$, 
        let $z'_j = \abs{z_j}$.
        Let \[
            s_j = 
            \begin{cases}
                0 & \text{ if $z_j \geq 0$} \\ 
                n^{d - 1} & \text{ if $z_j < 0$} \\ 
            \end{cases}
        \]
        Where $m = \max(z'_1, z'_2, z'_3)$
        , $n = z'_3 + 2$
        , $d = \floor{\log_{n}(m)} + 2$.
        \\
        Let $i = n^d$.
        Let 
        \[
            h(z_1, z_2, z_3) =
                [
                    i^3 + i^2 (z'_1 + s_1) + i^1 (z'_2 + s_2) + i^0 (z'_3 + s_3)
                    , n
                ]
        \]
        Where $h(z_1, z_2, z_3) \in \Z \times \Z$.

        Let $h^{-1}: \{h(a,b,c) | a,b,c \in \Z \} \to \Z \times \Z \times \Z$.
        Let $(z, n) \in \set{h(a,b,c) | a,b,c \in \Z}$, 
        where $n \geq 2$.
        Let $d = \floor{\log_n(z)} / 3$.
        Let $i \in \set{1,2,3}$,
        \[
            t_i = 
            \begin{cases}
                z & \text{ if $i = 3$} \\ 
                \floor{n^{-d} t_{i+1}} & \text{ otherwise} \\ 
            \end{cases}
        \]
        $a_i = t_i \mod n^{d - 1}$,
        and $s_i = t_i \mod n^{d}$.
        Let
        \[
            a_i' = 
            \begin{cases}
                a_i & \text{ if $a_i = s_i$} \\ 
                -a_i & \text{ otherwise} \\ 
            \end{cases}
        \]
        Let 
        \[
            h^{-1}(z,n) = (a_1', a_2', a_3')
        \]

        \newpage
        Proof that $h^{-1}(h(z_1,z_2,z_3)) = (z_1,z_2,z_3), \forall z_1,z_2,z_3 \in \Z$:\\ 
        $h^{-1}(h(z_1,z_2,z_3)) = h^{-1}(
            i^3 + i^2 (z'_1 + s_1) + i^1 (z'_2 + s_2) + i^0 (z'_3 + s_3)
            , n)
            $.
        In this case, $d' = \floor{\log_n(i^3 + i^2 (z'_1 + s_1) + i^1 (z'_2 + s_2) + i^0 (z'_3 + s_3))} / 3
            = 3d / 3 = d$.
        \footnote{$\forall j \in \set{1,2,3}, (z'_j + s_j) < i = n^d$ by construction in $h$.}
        Now, substituting these values into the functions in $h^{-1}$ we get:
        \[
            \begin{cases}
                t_3 = i^3 + i^2 (z'_1 + s_1) + i^1 (z'_2 + s_2) + i^0 (z'_3 + s_3) \\ 
                t_2 = i^2 + i^1 (z'_1 + s_1) + i^0 (z'_2 + s_2)  \\ 
                t_1 = i^1 + i^0 (z'_1 + s_1) \\ 
            \end{cases}
            \begin{cases}
                s_3' = i^0 (z'_3 + s_3) \\ 
                s_2' = z'_2 + s_2 \\ 
                s_1' = z'_1 + s_1 \\ 
            \end{cases}
            \begin{cases}
                a_3 = z'_3 \\ 
                a_2 = z'_2 \\ 
                a_1 = z'_1 \\ 
            \end{cases}
        \]
        From the construction of $z_j'$ and $s_j$ in $h$ we know: 
        \[
            z_{k} = 
            \begin{cases}
                z_{k}' & \text{ if $s_k = 0$ }\\ 
                -z_{k}' & \text{ if $s_k = 1$} \\ 
            \end{cases}
        \]
        Where $k = \set{1,2,3}$.
        And we know from $h^{-1}$ that:
        \[
            a_k' = 
            \begin{cases}
                a_k & \text{ if $a_k = s_k'$} \\ 
                -a_k & \text{ otherwise} \\ 
            \end{cases}
        \]
        Because $a_k = z_k'$ and $(a_k = s_k') \liff (z'_k = z'_k + s_k) \liff (s_k = 0)$,
        \footnote{And "otherwise" IFF $s_k = 1$, as theres only two cases.}
        so $z_k = a'_k$.
        Therefore \(
            h^{-1}(h(z_1,z_2,z_3)) 
            = h^{-1}(
                i^3 + i^2 (z'_1 + s_1) + i^1 (z'_2 + s_2) + i^0 (z'_3 + s_3)
                , n)
            = (a'_1, a'_2, a'_3)
            = (z_1,z_2,z_3)
        \)
        
        In fact, in this case, $n$ can be any function of $(z_1,z_2,z_3)$ where the output is an integer $\geq 2$.
        
        For example: For $n =  ((137 \times z_3) \mod 100003) + 2$: 
        \footnote{Using a Haskell script I wrote: \url{https://pastebin.com/nTEV7QvR}}
        \\ 
        $h(-631278, 432132, -1238790432) = (7110947828490755065940635335756984273816, 2094)$. \\ 
        $h^{-1}(7110947828490755065940635335756984273816, 2094) = (-631278, 432132, -1238790432)$.