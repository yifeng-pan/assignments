\section[Problem 4]{}
    \subsection[(i)]{List the even permutations in the symmetric group of degree $4$,
        i.e. the elements of the alternating group $A_4$. How many of them are of order $3$?}
        
        \begin{align*}
            p(\vectorvalue{1,2,3,4})
            \in \{
                &\vectorvalue{1,2,3,4},
                \vectorvalue{1,3,4,2},
                \vectorvalue{1,4,2,3},\\
                &\vectorvalue{2,1,4,3},
                \vectorvalue{2,3,1,4},
                \vectorvalue{2,4,3,1},\\
                &\vectorvalue{3,1,2,4},
                \vectorvalue{3,2,4,1},
                \vectorvalue{3,4,1,2},\\
                &\vectorvalue{4,1,3,2},
                \vectorvalue{4,2,1,3},
                \vectorvalue{4,3,2,1}\}
        \end{align*}
        \footnote{Source: \url{https://groupprops.subwiki.org/wiki/Element_structure_of_alternating_group:A4}}
        There are $8$ of order $3$.

    \subsection[(ii)]{Let $G$ be a group. Show that a subgroup $H$ of $G$ of index $2$ is necessarily normal.}
        \footnote{Source: \url{https://proofwiki.org/wiki/Subgroup_of_Index_2_is_Normal}}
        Let $H$ be a subgroup of $G$, and $\abs{G:H} = 2$.

        Since the index is $2$, partition $G$ into $H, H'$, 
        so $H \union H' = G$, and $H \intersection H' = \emptyset$.

        Let $g \in G$.

        If $g \in H$, then $gH = Hg$, and we're done.

        Suppose $g \in H'$.
        Then $gH = H'$, otherwise $H = H'$.
        Similarly $Hg = H'$.
        Therefore $gH = H' = Hg$.

        Therefore $H$ is a normal subgroup of $G$.
        \qed


    \subsection[(iii)]{Let $K$ be a subgroup of $A_4$ of order $6$. Show that for all $a \in A_4$,
        the cosets $K, aK,$ and $a^2K$ cannot all be distinct, and deduce that $K$ must necessarily
        contain all elements of order $3$ of $A_4$. Conclude that $A_4$ does not have a subgroup of 
        order $6$, even though $6$ divides the order of $A_4$.}
        
        \footnote{Source: \url{https://math.stackexchange.com/questions/582658/a-4-has-no-subgroup-of-order-6}}
        Let $K$ be a subgroup of $A_4$.
        Let $\abs{K} = 6$.

        % Since $\abs{G:K} = 2$, $K$ is normal.
        

        If $K, aK, a^2K$ are all distinct, then $\abs{G:H} \geq 3$, but $\abs{G:H} = 2$.
        Therefore $K, aK, a^2K$ are not all distinct.
        \qed

        Since there are $8$ elements of order $3$ in $A_4$,
        and $K$ only contains $6$ elements,
        choose $a \in A_4, a \not\in K$, and the order of $a = 3$.

        Suppose $K = aK$.
        Since $1 \in K, a = a1,$ and $a \in K$.
        Contradiction.

        Suppose $K = a^2K$.
        Similarly $a^2 \in K$, but $K$ is a group, so ${(a^2)}^{-1} = a \in K$.
        Contradiction.

        Suppose $aK = a^2K$.
        Similarly $a(a) = a^2 (1) $, so $a \in K$.
        Contradiction.

        Therefore $K$ must necessarily contain all $a\in A_4$ of order $3$.\\
        Therefore the subgroup $K$ of $A_4$ such that $\abs{K}= 6$ does not exist.
        \qed