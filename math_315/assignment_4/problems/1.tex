\section[Problem 1]{Let $R$ be a commutative ring and let $I$ and $J$ be ideals of $R$.}
    \subsection[(i)]{Prove that the sets $I + J = \set{x + y | x\in I, y \in J}, I \intersection J$
        and \\
        \(
            IJ = 
            \set{x_1y_1+ x_2y_2 +\hdots + x_n y_n 
            | n \geq 1, x_m \in I, y_m \in J, \lall 1 \leq m \leq n}
        \) 
        are ideals of $R$.
    }
        \begin{multicols}{3}
            \subsubsection[Addition]{
                \footnote{Reference: “Algebra” by Michael Artin, Section 11.3, Homomorphisms and Ideals.}
                $I + J$:
            }
                % $I+J$:

                Its clear that $I+J \neq \emptyset$.

                Let $x +y, x'+y' \in I+J$.
                Now, $x + y + x' + y' = x+x' + y+y' \in I+J$.

                Let $x + y \in I+J, r \in R$.
                Now, $r(x+y) = rx + ry \in I +J$.

                Therefore $I+J$ is an ideal.
                \qed
            
            \subsubsection[Intersection]{$I \intersection J$:}
                % $I\intersection J$:

                $\set{0} \subseteq I \intersection J$.

                Let $x, x' \in I \intersection J$.
                Now, $x + x'$ is closed under $I$ and under $J$, therefore $x+x' \in I \intersection J$.

                Let $x \in I+J, r \in R$.
                Now, $rx \in I$ and $rx \in J$, \\
                therefore $rx \in I \intersection J$.
                
                Therefore $I \intersection J$ is an ideal.
                \qed

            \subsubsection[Linear Combination]{$IJ$:}
                % $IJ$:

                Its clear that $IJ \neq \emptyset$.

                Let $z, z' \in IJ$.
                Now, $z + z' = x_1y_1 + \hdots + x_ny_n + x_1'y_1' + \hdots + x_{n'}'y_{n'}' \in IJ$.

                Let $z \in IJ, r \in R$.
                Now, $rz = r(x_1y_1 + \hdots + x_ny_n) = (rx_1)y_1 + \hdots + (rx_n)y_n$.
                Since $\lall k, rx_k \in I$, so $rz \in IJ$.

                Therefore $IJ$ is a ideal.
                \qed
        \end{multicols}

    \subsection[(ii)]{Show that $IJ \subset I \intersection J$, and prove that if
        $I+J = R$, then $IJ = I \intersection J$.}
        % \na
        Let $z \in IJ$, where $z = x_1y_1+ x_2y_2 +\hdots + x_n y_n$, $x_k \in I, y_k \in J, \lall k$.

        Since $I$ is an ideal, we know $x_k y_k \in I$, $\lall k$.
        And since $I$ is closed under addition, $z \in I$.

        Simularly $z \in J$. Therefore $z \in I \intersection J$, and $IJ \subseteq I\intersection J$.
        \qed

        Suppose $I + J = R$.
        
        \footnote{Reference: \url{https://en.wikipedia.org/wiki/Ideal_(ring_theory)\#Ideal_operations}.}
        $I \intersection J = R (I \intersection J) 
            = (I + J) (I \intersection J)
            = I (I\intersection J) + J (I \intersection J)$

            Since $I \intersection J \subseteq J$,
            we have $I (I\intersection J) + J (I \intersection J)
            \subseteq I J + J (I \intersection J)
            \subseteq I J + J I
            = IJ
        $.\qed
        
        Therefore $IJ = I \intersection J$.


    \subsection[(iii)]{Let $a$ and $b$ be relatively prime integers. Prove that 
        there are integers $m,n$ such that $a^m+b^n = 1$ modulo $ab$.
        (i.e. $a^m + b^n = 1$ in $\Z/ab\Z$. Note that this makes sense for $ab < 0$ as well.)
    }
        \footnote{Reference: \url{https://math.stackexchange.com/questions/901559/am-bn-equiv-1-mod-ab-for-some-m-n}.}
        \footnote{\url{https://en.wikipedia.org/wiki/Chinese_remainder_theorem\#Theorem_statement}.}
        By the Chinese Remainder Theorem: 
        We define a ring isomorphism from $\Z/ab\Z \sim \Z/a\Z \times \Z/b\Z$.

        It is sufficient to prove that $\lis m,n \in \Z$ such that 
        $a^m+b^n \cmod{1}{a}$, and $a^m+b^n \cmod{1}{b}$.

        Or equivalently: $b^n \cmod{1}{a}$, and $a^m \cmod{1}{b}$.

        \footnote{\url{https://en.wikipedia.org/wiki/Euler's_theorem}}
        By Euler's Theorem:
        Since $a$ and $b$ are coprimes,
        both $m$ and $n$ exists.
        \qed