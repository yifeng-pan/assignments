\section[Problem 3]{}
    \subsection[(i)]{Prove that in the ring $\Z[x]$ of polynomials over the integers,
        the intersection of the principal ideals generated by $2$ and $x$ is the principal ideal
        generated by $2x$ i.e. $(x)\intersection(2) = (2x)$.
    }
        % \na
        $(x)$ contains all the multiples of $x$. Or $(x) = \set{z x | z \in Z[x]}$.

        $(2)$ contains all the multiples of $2$.

        $(x) \intersection (2)$ contians all multiples of $2$ and $x$.

        $(2x)$ contain all the multiples of $2x$.

        $(x) \intersection (2) = (2x)$.
        \qed
        % Let $s \in (x) \intersection (2)$.


        % Where 
        % $\set{z x | z \in Z[x]} \intersection \set{z 2 | z \in Z[x]}
        % = \set{z 2x | z \in Z[x]}$.


    \subsection[(ii)]{Which principal ideals of $\Z[x]$ are maximal ideals? Justify your answers.}
        \footnote{Reference: Artin's Algebra, Section 11.8, Maximal Ideals.}
        \footnote{Reference: Third Example in \url{https://en.wikipedia.org/wiki/Maximal_ideal\#Examples}.}
        % \footnote{\url{https://minhyongkim.files.wordpress.com/2011/10/maxideal.pdf}}
        Non.

        Let $f(x) \in \Z[x]$ such that $(f(x))$ is a principal maximal ideal.

        % Let $\set{c_k}$ be the coffiecients of $f(x)$.

        Suppose $f(x)$ is not a constant.

        % % Find a number $n > 1$ such that $n$ is co-prime to every element of $\set{c_k}$. 
        % $n$ exists since $\set{c_k}$ is finite.

        Since $2 \not\in (f(x))$, we have $(f(x)) \subsetneq (f(x), 2)$.

        Now, we need to prove $(f(x), 2) \neq \Z[x]$.

        It is sufficient to prove $\lall a,b \in \Z[x], af(x) + 2b \neq 1$.

        $af(x) + 2b =1 \iff af(x)$ is an odd integer.
        But $f(x)$ is not a constant, so $af(x)$ cannot be a non-zero integer.
        We're done.

        Now suppose $f(x) = c \neq 1$ is an integer. If $f(x) = 1$, then $(f(x))$ is not a maximal ideal.

        Since $x \not\in (c)$, we have $(c) \subsetneq (c, x)$.

        It's analogous to verify that $1 \not\in (c,k)$.
        We're done.

        Therefore we can construct a proper ideal that is a strict superset of $(f(x))$ in both cases.
        And $(f(x))$ is not a maximal ideal.
        \qed


        
    \subsection[(iii)]{Let $\R$ be the field of real numbers. What are the maximal ideals of the factor
        ring $\R[x]/(x^2)$? Justify your answer.}

        Lemma: $\R[x]$ and $\R[x]/(x^2)$ are both principal ideal domains.

        Let $\phi: \R[x] \to \R[x]/(x^2)$ be a surjective ring homomorphism,
        where $\ker{\phi} = (x^2)$.

        The ideals of $\R[x]$ that contains $(x^2)$ are $(1),(x),$ and $(x^2)$.

        \footnote{
            This is also stated in Artin's Algebra as Theorem 11.4.3: \\
            \url{https://proofwiki.org/wiki/Correspondence_Theorem_for_Ring_Epimorphisms}
        }
        By the Correspondence Theorem: The only ideals in $\R[x] / (x^2)$ are $(1), (x^2)=(0),$ and $(x)$.

        Therefore $(x)$ is the only proper ideal,
        and the only maximal ideal.
        \qed
        

        % $\R[x]/(x^2)$ is the factor ring containing all polynomials of degree $\leq 1$.

        % Assuming its a principal ideal field.

        % Non are generated by constants.

        % Non are generated by invertable polynomials.

        % Let $ax + b$ be a poly.

        % $(ax + b)(cx + d) = 1$.

        % So $ac + bd = 1, ad + bc = 0$.