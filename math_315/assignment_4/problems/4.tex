\section[Problem 4]{Let $R$ be a commutative ring.}
    \subsection[(i)]{Assume that the characteristic of $R$ is a prime number $p$.
        Prove that if $r \in R$ is nilpotent i.e. $r^n = 0$ for some $n \geq 0$,
        then there is an $l \geq 1$ such that $(1 + r)^l= 1$.
    }
        % \footnote{\url{https://en.wikipedia.org/wiki/Frobenius_homomorphism}}
        % \footnote{Reference: \url{https://en.wikipedia.org/wiki/Characteristic_(algebra)}.}
        Since the characteristic of $R$ is $p$,
        $\lall r \in R, p r = 0$.
        
        Let $r \in R$.

        % \footnote{Proof: 
        %     \url{https://math.stackexchange.com/questions/2834493/prime-number-rows-in-a-pascals-triangle/2834502\#2834502}.}
        \footnote{\url{https://en.wikipedia.org/wiki/Freshman\%27s_dream\#Prime_characteristic}.}
        We know $(1+r)^p = 1 + r^p$.
        And $(1+r)^{(p^k)} = 1+r^{(p^k)}$.

        \footnote{I got the idea that I needed $p^k \geq n$, not $p^k \cmod{0}{n}$,
            from Devin Kwok (UCID: 10016484).}
        Choose $k$ such that $p^k \geq n$.

        Let $l = p^k$.

        Then $(1+r)^l = 1 + r^l = 1 + r^nr^{l-n} = 1 + 0r^{l-n} = 1$.
        \qed

%         By Euler's Theorem:
%         Since $p$ and $n$ are coprime,
%         $\lis k$ such that $p^k \cmod{1}{n}$.

%         This means $(1 + r)^{(p^k)} = 1 + r^{(p^k)} = 1 + r$.

% (
%         Doesn't work.
%         R is not an integral domain.
% )

%         So $(1 + r)^{(p^k)} - (1 + r) = 0
%         \to (1 + r) ( (1 + r)^{(p^k) - 1} - 1) = 0$.

%         Therefore $(1 + r)^{(p^k) - 1} = 1$.
%         Let $l = p^k - 1$.
        % \qed


    \subsection[(ii)]{Prove that an integral domain of finite order is a field.}
        \footnote{Reference: \url{https://en.wikipedia.org/wiki/Integral_domain}.}
        Let $R = \set{0, r_1, r_2, \hdots, r_{n-1}}$ be an finite integral domain of order $n$.

        We prove the only ideals of $R$ are $(0)$ and $(1)$.

        Let $I$ be an ideal of $R$.

        Suppose $I \neq (0)$.

        Let $i \in I, i \neq 0$.

        Let $J = (i) = \set{ir | r \in R} = \set{0} \union \set{ir_1, ir_2, \hdots, ir_{n-1}}$.

        We prove $ir_1, ir_2, \hdots, ir_{n-1}$ are distinct.

        Suppose $\lis a,b, a \neq b$ such that $ir_a = ir_b$.

        Now, $ir_a = ir_b \to 0 = ir_a - ir_b = i(r_a - r_b)$.

        Since $R$ is an integral domain, and $i \neq 0$, we know
        $r_a - r_b = 0$. Contradiction.

        Therefore $ir_1, ir_2, \hdots, ir_{n-1}$ are distinct,
        and the order of $J$ is $1 + (n - 1) = n$.

        Therefore $I = J = R$.

        Therefore $R$ only has two ideals.

        \footnote{Reference: “Algebra” by Michael Artin, Proposition 11.3.19 (b).}
        Therefore $R$ is a field.
        \qed

    \subsection[(iii)]{Find $x \in \Z$ such that $x = 3$ modulo $8$ and $x = 2$ modulo $5$.}
        \footnote{Reference: \url{https://en.wikipedia.org/wiki/Chinese_remainder_theorem\#Computation}.}
        $5$ and $8$ are coprimes.

        So we only have to check integers between $0$ and $5 * 8 = 45$.

        The solutions to $x \cmod{2}{5}$ are $\set{2,7,12,17,22,27,32,37,42}$.

        The solutions to $x \cmod{3}{8}$ are $\set{3, 11, 19, 27, \hdots}$.

        We found $x = 27$.