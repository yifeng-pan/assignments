\section[Problem 4]{
    Recall the group 
    \(
        GL_2(\R) =
        \left \{
        \begin{bmatrix}
            a & b \\
            c & d
        \end{bmatrix}
        \middle |
        a, b, c, d \in \R
        \text{ and }
        ad - bc \neq 0
        \right \}
    \),
    with matrix multiplication.
}
    \subsection[(i)]{
        Show that the subset 
        \(
            \left \{
            \begin{bmatrix}
                a & b \\
                c & d
            \end{bmatrix}
            \middle |
            a, b, c, d \in \Z
            \text{ and }
            ad - bc \neq 0
            \right \}
        \)
        is not a subgroup.
    }
        Let $H$ be the above subset.
        Let \(
            a = 
            \begin{bmatrix}
                2 & 0 \\
                0 & 2
            \end{bmatrix}
            , a \in H
        \).
        So \(
            a^{-1} = 
            \begin{bmatrix}
                \frac{1}{2} & 0 \\
                0 & \frac{1}{2}
            \end{bmatrix}
            , a^{-1} \not\in H
        \).
        Therefore $H$ is not a group, and therefore not a subgroup.

    \subsection[(ii)]{
        Show that the subset $SL_2(\Z)$
        of invertible matrices with integer coefficients and determinant $1$,
        i.e.
        \[
            \left \{
            \begin{bmatrix}
                a & b \\
                c & d
            \end{bmatrix}
            \middle |
            a, b, c, d \in \Z
            \text{ and }
            ad - bc = 1
            \right \}
        \]
        is a subgroup.
    }
        We know $I_2 \in SL_2(\Z) \subseteq GL_2(\R)$ and matrix multiplication is associative.

        Let $A,B \in SL_2(\Z)$.
        Now, $\det(AB) = \det(A)\det(B) = 1 \times 1 = 1$.
        Since $AB$ is an integer matrix, $AB \in SL_2(\Z)$.
        Therefore $SL_2(\Z)$ is closed.

        % Let $A \in SL_2(\Z)$.
        % Since $\det(A) \neq 0$,
        % $A^{-1}$ exists,
        % and $\det(A^{-1}) = \dfrac{1}{\det(A)} = 1$.

        Let \(
            A = 
            \begin{bmatrix}
                a & b \\
                c & d
            \end{bmatrix}
            \in SL_2(\Z)
        \).
        We know \( 
            A^{-1} =
            \dfrac{1}{ad-bc}
            \begin{bmatrix}
                d & -b \\
                -c & a 
            \end{bmatrix}
            =
            \begin{bmatrix}
                d & -b \\
                -c & a 
            \end{bmatrix}
        \),
        which is an integer matrix with $\det(A^{-1}) = da - (-c)(-b) = ad - bc = 1$.
        Therefore $A^{-1}$ exists and is in $SL_2(\Z)$.

        Therefore $SL_2(\Z)$ is a group and a subgroup of $GL_2(\R)$.


    \subsection[(iii)]{
        Let $SL_2(\Z / 3\Z)$ be $SL_2(\Z)$ with the matrix entries interpreted modulo $3$.
        It is a group.
        What is the order of \(
            \begin{bmatrix}
                1 & 1 \\
                0 & 1 
            \end{bmatrix}
            \in SL_2(\Z/3\Z)
        \)?
    }
        \(
            \begin{bmatrix}
                1 & 1 \\
                0 & 1 
            \end{bmatrix}^1
            =
            \begin{bmatrix}
                1 & 1 \\
                0 & 1 
            \end{bmatrix}
            \neq I_2
        \) and
        \(
            \begin{bmatrix}
                1 & 1 \\
                0 & 1 
            \end{bmatrix}^2
            =
            \begin{bmatrix}
                1 & 2 \\
                0 & 1 
            \end{bmatrix}
            \neq I_2
        \) and
        \(
            \begin{bmatrix}
                1 & 1 \\
                0 & 1 
            \end{bmatrix}^3
            =
            \begin{bmatrix}
                1 & 0 \\
                0 & 1 
            \end{bmatrix}
            = I_2
        \) under $SL_2(\Z/3\Z)$.

        Therefore the peroid of \(
            \begin{bmatrix}
                1 & 1 \\
                0 & 1 
            \end{bmatrix}
            \in SL_2(Z/3\Z)
        \) is $3$,
        as $3$ is the smallest positive integer with this property.
        \qed
        
        % The order of the group $SL_2(\Z / 3\Z)$ 
        % is the number of solutions to $a,b,c,d \in \set{0,1,2}, ab - cd = 1$.

        % Solutions: \(
        %     \set{
        %         (1,1,0,0), 
        %         (1,1,0,1), 
        %         (1,1,0,2), 
        %         (1,1,1,0), 
        %         (1,1,2,0), 
        %         (1,2,1,1), 
        %         (2,1,1,1)
        %     }
        % \).

        % $\abs{SL_2(\Z / 3\Z)} = 7$.
        % \qed