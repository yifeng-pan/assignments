\section[Problem 2]{Let $p$ be a prime number.
}  
    \subsection[(i)]{Determine the group of group automorphisms of $(\Z/p\Z, +)$.
    }
        Let $A$ be the group of group automorphisms of $(\Z/p\Z, +)$.

        Let $g$ be a generator of $(\Z/p\Z, +)$.

        Since automorphisms are ismorphic homomorphisms, an automorphism of $(\Z/p\Z, +)$ 
        must map $g$ to a generator.

        Since $p$ is prime, $(\Z/p\Z)$ has $p-1$ unique generators.
        Therefore $\abs{A} = p-1$,
        and the elements of $A$ would be the unique automorphisms defined by 
        \begin{align*}
            \phi_1(g) = g, \phi_2(g) = g^2, \phi_3(g) = g^3, \hdots, \phi_{p-1}(g) = g^{p-1}
        \end{align*}
        where $g^n = \sum_{i=1}^n g$.

    \subsection[(ii)]{Determine the group of ring automorphisms of $\Z/p\Z$, for the ring $\Z/p\Z$
        with the usual addition and multiplication of integers modulo $p$.
    }
        Let $A$ be the group of ring automorphisms of $\Z/p\Z$.

        Similarly to part (i), we consider the generators of $\Z/p\Z$.

        Since every generator of $(\Z/p\Z, \times)$ is a generator of $(\Z/p\Z, +)$,
        we only have to consider the case for $(\Z/p\Z, \times)$.

        Let $\set{g_1, g_2, \hdots}$ be the generators/primative roots of $(\Z/p\Z, \times)$.

        Let $n$ be the number of primative roots of $(\Z/p\Z, \times)$.

        Then $\abs{A} = n$, and the elements of $A$ would be the unique automorphisms defined by 
        \[
            \phi_1(g_1) = g_1, \phi_2(g_1) = g_2, \phi_3(g_1) = g_3, \hdots, \phi_{n}(g_1) = g_n
        \]

    \subsection[(iii)]{Determine the group of automorphisms of the symmetric group $S_3$ of permutations of
        the set $\set{1,2,3}$.
    }
        An automorphism of $S_3$ would map some permutation of $\set{1,2,3}$ to some permutation.

        Therefore the group of group automorphisms of $S_3$ is $S_3$ itself.