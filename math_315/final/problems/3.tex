\section[Problem 3]{
}  
    \subsection[(i)]{
        Define the class equation of a finite group $G$.
    }
        \textbf{Definition 1:} (Used in (3.2))\\
        Define an equivalence relation of $G$ such that $a \sim b \iff \lis g \in G, gag^{-1} = b$.

        The class equation is the summation of the sizes of the equivalence classes of the above relation.

        \textbf{Definition 2:} (Used in (3.3))\\
        Define a group action of $G$ on itself such that $(g,x) \to gxg^{-1}$.

        Denote an orbit of this action $C(x)$ as an conjugacy class.
        Denote an stabilizer subgroup of this action $Z(x)$ as an centralizer.

        The class equation is $\abs{G} = \sum \abs{C(x)}$.


    \subsection[(ii)]{
        What is the class equation of an abelian group of order $10$? Justify your answer.
    }
        Using the equivalence relation from above.

        Since the binary operation of an abelian group is commutative, $gag^{-1} = b \iff a = b$.
        Therefore the class equation is $10 = \sum_{i=1}^{10} 1$, 
        since the group has $10$ distinct elements. 

    \subsection[(iii)]{
        Show that there is no group of class equation $10 = 1+1+1+1+1+5$.
    }   
        Let $G$ be a group with the class equation 
        $10 = 1+1+1+1+1+5 = \abs{C_1} + \abs{C_2}+ \hdots + \abs{C_6}$.

        Since there are $5$ conjugacy clases with the size $1$,
        these $5$ elements are commutative with every element of $G$.
        Therefore the center of $G$ contains these $5$ elements.

        % Center of $G = C_1 \union C_2 \union C_3 \union C_4 \union C_5$.

        Now, we find the size of the centralizers using the equlity: $\abs{G} = \abs{C(x)} \abs{Z(x)}$.\\
        We get $\abs{Z_1} = \abs{Z_2} = \abs{Z_3} = \abs{Z_4} = \abs{Z_5} =  10 , \abs{Z_6} = 2$.

        Now, since the center of $G$ is also the intersection of all the centralizers of $G$,
        and $\abs{Z_6} = 2$, then order of the center $\leq 2$.

        But the center contains $5$ elements, therefore we have a contradiction, and there exists no such 
        group $G$.
        \qed


