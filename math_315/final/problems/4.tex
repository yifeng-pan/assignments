\section[Problem 4]{
}  
    \subsection[(i)]{
        Describe the units in the direct product ring $M_2(\R) \times M_2(\R)$,
        where $M_2(\R)$ is the ring of $2$ by $2$ matrices with real coefficients.
    }
        To find the units of $R = M_2(\R) \times M_2(\R)$,
        we need to find the elements of $R$ that has a multiplicative inverse.

        The multiplicative identity $R$ is \(
            (I_2, I_2)
        \), where $I_2$ is the $2$ by $2$ identity matrix.

        The units are therefore: $\set{(A,B) | A, B \in M_2(\R), \det(A) \neq 0, \det(B) \neq 0}$.
        \qed

    \subsection[(ii)]{
        Is the direct product of groups $\Z/7\Z \times \Z/11\Z$ a cyclic group?
        Justify your answer.
    }
        If $\Z/7\Z$ and $\Z/11\Z$ are under multiplication,
        then their orders would be $6$ and $10$ respectively.
        $6$ and $10$ are not relatively prime, and their lowest common multiple $= 30$.

        Therefore a ``generator'' of $\Z/7\Z \times \Z/11\Z$ can only ever cover $30$ elements.
        But $\Z/7\Z \times \Z/11\Z$ has $6 * 10 = 60$ elements. \\
        \textbf{Therefore $\Z/7\Z \times \Z/11\Z$ is not a cyclic group under multiplication,
        since there is no generator for it.}
        \qed

        If $\Z/7\Z$ and $\Z/11\Z$ are under addition,
        then their orders would be $7$ and $11$ respectively.
        $7$ and $11$ are relatively prime, and their lowest common multiple $= 7*11 = 77$.

        Let $G = \Z/7\Z \times \Z/11\Z$ under addition.
        We prove $(1,1)$ is a generator.
        Since the period of $1$ in $\Z/7\Z$ is $7$,
        and the period of $1$ in $\Z/11\Z$ is $11$,
        the period of $(1,1)$ in $G$ is their lowest common multiple $77$.

        But $G$ only contains $77$ elements, so $(1,1)$ is a generator.
        \textbf{Therefore $\bm{G}$ under addition is a cyclic group.}
        \qed