\section[Problem 6]{
    Let $p$ be a prime number.
}  
    \subsection[(i)]{
        Show that there are $\frac{p(p+1)}{2}$ reducible monic quadratic polynomials over
        the finite field $\F_p = \Z/p\Z$.
    }
        Monic quadratic polynomials are of the form $x^2 + ax + b$, for some $c,d\in \F_p$.

        There are $p^2$ polynomials of this form in $\F_p$.

        Since quadratics are of degree $2$, the are either irreducible,
        or can be reduced into two polynomials with degree $1$.

        Therefore the reducible polynomials are of the form 
        $x^2 + ax + b = (x + c)(x + d) = x^2 + (c+d)x + cd$, for some $c,d \in \F_p$.

        % The number of distinct reducible polynomials are therefore the number of distinct
        % solutions to the equations $c+d = a, cd = b$, $a,b,c,d \in \F_p$.

        The number of distinct reducible polynomials are therefore the number of distinct
        images of $(x+c)(x+d)$ with $c,d \in \F_p$.

        We count:
        \begin{enumerate}
            \item Choose $c \in \F_p$. ($p$ choices)
            \item Choose $d \in \F_p, d \neq c$. ($p - 1$ choices)
            \item The choices for $(c,d)$ are commutative, so we counted everything twice. ($1/2$)
            \item Now we add the the number of ways to choose $c,d\in \F_p, c= d$. ($p$ choices).
        \end{enumerate}

        We have $\frac{p(p-1)}{2} + p = \frac{p^2 - p + 2p}{2} = \bm{\frac{p(p+1)}{2}}$.
        \qed





    \subsection[(ii)]{
        Construct a field of order $49$.
    }
        $F = \F_7 / (x^2 + 1)$.

        Since $x^2 + 1$ has degree $2$, if it is reducible, it can be reduced into
        two polynomials of degree $1$, which would imply it has integer roots.

        It's easy to check that $x^2 + 1 \neq 0, \lall x \in \set{0,1,2,3,4,5,6} = \F_7$.
        Therefore $x^2 + 1 $ is irreducible in $\F_7$.

        Since $x^2 + 1$ is irreducible in $\F_7$, $F$ is a field.
        \footnote{I proved this in Assignment 5, 
            I've provided a copy of the proof on the next page.}

        Since $x^2 + 1$ kills all the polynomials of above degree $2$ in $\F_7$,
        all the elements of $F$ are of the form $ax + b, a,b \in \F_7$.

        Therefore $\abs{F} = 7^2 = 49$.
        \qed

    \subsection[(iii)]{
        Is there a field $F$ of characteristic $p$ such that $F$ is an infinite set?
        Prove that $F$ does not exist or give an example.
    }
        We know $\Z/2\Z[x]$ is an integral domain.

        Define $F$ as a field of fractions over $\Z/2\Z[x]$.

        $F$ is a field by construction.

        We have to prove $F$ is infinite with a characteristic $2$.

        Let $[\frac{a}{b}]\in F$, for $a,b\in \Z/2\Z[x], b \neq 0$.

        $[\frac{a}{b}] + [\frac{a}{b}] = [\frac{2a}{b}] = [\frac{0}{b}]$, since $2a = 0 \in \Z/2\Z[x]$,
        where $[\frac{0}{b}]$ is the additive identity in $F$.

        Therefore $F$ has a characteristic of $2$.

        Now, we construct an infinite subset of $F$.
        Let $S = \set{[\frac{x}{1}],[\frac{x^2}{1}],[\frac{x^3}{1}],[\frac{x^4}{1}],\hdots} \subseteq F$.

        The elements of $S$ are all distinct equivalence classes, i.e. elements of $F$,
        since $x^a 1 \neq x^b 1, \lall a,b\in \N, a\neq b$.

        Since $S$ is denumerable,
        $F$ is not finite.
        \qed



\section*{Lemma used in (6.2)}
    \begin{lemma}
        $\F_p[x] / (f(x))$ is a field $\iff$ $f(x)$ is irreducible in $\F_p[x]$.
    \end{lemma}
        Proof:
        Suppose $f(x) \in \F_p[x]$ is irreducible (henceforth refered to as $f$).

        Let $[g] \in \F_p[x]/(f)$, where $g \in \F_p[x] \setminus \set{0}$.
        We find the inverse of $[g]$.

        Since $f$ is irreducible, $\gcd(f, g) = c$ for some constant $c \in \F_p$.
        Otherwise, $c$ would be a non-constant factor of $f$.

        By Bézout's Identity: \\
        $\lis a, b \in \F_p[x]$, such that 
        $af + bg = 1$.
        Now, 
        \begin{align*}
            [af + bg] &= [1] \\
            % [af] + [bg] &= [1] \\
            [a][f] + [b][g] &= [1] \\
            [a][0] + [b][g] &= [1] \\
            % af + bg + (f) &= 1 + (f) \\
            % bg + (f) &= 1 + (f) \\
            [b][g] &= [1]
        \end{align*}
        Therefore $[b]$ is the multiplicative inverse of $[g]$ in $\F_p[x] / (f)$,
        where $[1]$ is the multiplicative identity.

        Since $\F_p[x] / (f)$ is a quotient ring, and since every element has an inverse,
        $\F_p[x] / (f)$ is a field.

        Now for the logical inverse: Suppose $f$ is reducible.
        So $\lis$ non-constants $a,b \in F_p[x]$ such that $ab = f$.
        
        We have
        \(
            [ab] = [f] \to
            [a][b] = [0]
        \).

        Suppose $[a] = [0]$.
        So $\lis r \in \F_p[x]$, such that $rf = a$.
        Now, $ab = f \to rfb = f$,
        and since we know $\F_p[x]$ is an integral domain, $rb = 0$.
        But $r \neq 0$, $b \neq 0$ from construction
        ($r\neq 0$ because $a \neq 0$)
        ,so $\F_p[x]$ has non-zero zero divisors.
        Contradiction.

        Therefore $[a] \neq [0]$.
        Similarly, $[b] \neq [0]$.

        Since $[a] \neq [0]$ and $[b] \neq [0]$, $\F_p[x] / (f)$ has non-zero zero divisors,
        and therefore is not an integral domain, 
        and therefore not a field.
        \qed
