\section[Problem 3]{}
    \subsection[(i)]{Let $R, R'$ be rings. Show that the product of groups $R\times R'$
        is a ring for the multiplication given by $(r_1, r_1') \dotp (r_2, r_2') = (r_1r_2, r_1'r_2')$.
        It is called the (direct) product of the rings $R$ and $R'$.}

        It's easy to verify that $(R \times R', +)$ is an abelian group with the identity $(0,0)$.

        It's also easy to verify that $\times$ is commutative and associative on $R\times R'$, 
        with the identity being $(1,1)$.

        Let $(a, a'), (b,b'), (c,c') \in R\times R'$.
        Now, \begin{align*}
            ((a,a') + (b,b')) (c, c')
            &= (a+b, a'+b') (c,c') \\
            &= ((a+b)c, (a'+b')c') \\
            &= (ac+bc, a'c'+b'c') \\
            &= (a,a')(c,c') + (b,b')(c,c')
        \end{align*}
        Therefore the distributive law holds, and
        $R\times R'$ is a ring.
        \qed

    \subsection[(ii)]{Prove that a surjective map $f: R \to R'$ such that $f(r_1r_2) = f(r_1)f(r_2)$
        for all $r_1, r_2 \in R$ satisfies $f(1) = 1$.}
        Let $b \in R'$.

        Since $f$ is surjective, $\lis a \in R, f(a) = b$.

        Now, $f(1)b = f(1)f(a) = f(1a) = f(a) = b$, and $bf(1) = b$.
        
        Therefore $f(1)$ is the identity in $R'$.
        \qed

    \subsection[(iii)]{An element $r$ of a ring $R$ is nilpotent if $r^n = 0$ for some $n \geq 0$.
        Does the ring $F(\R, \R)$ of functions from $\R$ to $\R$ have non-zero nilpotent elements?
        Does it have non-zero zero divisors? Prove or give an example.}

        \newcommand\threec{F(\R,\R)}

        Lemma: $F(\R,\R)$ of all function from $\R$ to $\R$, is a ring.

        There are no non-zero nilpotent elements. Proof:
        
        Let $f \in \threec$, such that $f(x) \neq 0$.

        So $\lis x \in \R$, such that $f(x) = c \neq 0$, for some $c \in \R$.

        So $(f(x))^n = c^n \neq 0$ for all $n \in \N$.
        \qed

        There are non-zero zero divisors. Proof:

        Let $f(x) = 
        \begin{cases}
            0 &\text{ if $x = 1$} \\
            1 &\text{ if $x \neq 1$} 
        \end{cases}$.

        Let $g(x) = 
        \begin{cases}
            1 &\text{ if $x = 1$} \\
            0 &\text{ if $x \neq 1$} 
        \end{cases}$.

        Where $f(x) \neq 0, g(x) \neq 0, f(x)g(x) = 0$.
        \qed