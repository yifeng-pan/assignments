\section[Problem 2]{}
    \subsection[(i)]{Using the division algorithm, find the greatest common divisor of $x^4 + x + 1$
        and $x^3 + 2$ in $\Q[x]$ and express it as a linear combination of these polynomials. 
        Explain your computation.}
        % \footnote{Reference: \url{https://en.wikipedia.org/wiki/Polynomial_long_division}}
        \begin{center}
            \begin{tabular}{|c|c|c|c|c|}
                \hline
                $\bm{i}$& $\bm{r}$          & $\bm{s}$  & $\bm{t}$  & $\bm{q} = r_{i-1} / r_i$ \\
                \hline
                $0$ & $x^4 + x + 1$                         & $1$  & $0$  & \----  \\
                \hline
                $1$ & $x^3 + 2$                             & $0$  & $1$  & $x$    \\
                \hline
                $2$ & $(x^4 + x + 1) - x(x^3 + 2) = -x+1$   & $1$  & $-x$  & $-x^2$ \\
                \hline
                $3$ & $(x^3 + 2) -(- x^2)(-x+1) = x^2 + 2$  & $x^2$& $1-x^3$  & $0$    \\
                \hline
                $4$ & $(-x+1) -0(x^2 + 2) = -x+1$           & $1$  & $-x$  & $-x$   \\
                \hline
                $5$ & $(x^2+2) - (-x)(-x+1) = x+2$          & $x^2+x$  & $1-x^3-x^2$  & $-1$   \\
                \hline
                $6$ & $(-x+1) - (-1)(x+2) = 3$              & $1+x^2+x$  & $1-x^3-x^2-x$   & $(x+2)/3$  \\
                \hline
                $7$ & $(x+2) - ((x+2)/3)(3) = 0$            & $s_5 - s_6q_6$  & $t_5 - t_6q_6$  & \----  \\
                \hline
            \end{tabular}
        \end{center}
        
        \footnote{Reference: \url{https://en.wikipedia.org/wiki/Polynomial_greatest_common_divisor\#B\%C3\%A9zout's_identity_and_extended_GCD_algorithm}}
        where
        $s_5 - s_6q_6 = (x^2+x) - \frac{x+2}{3}(1+x^2+x) = \frac{- x^3 - 2}{3}$, 
        \footnote{Calculator: \url{https://www.symbolab.com/}.}
        \\
        and 
        $t_5 - t_6q_6 = (1-x^3-x^2) - \frac{x+2}{3}(1-x^3-x^2 - x) = \frac{x^4+x+1}{3}$.

        $r_6 = 3 = r_0s_6 + r_1t_6 = (x^4+x+1)(1+x^2+x) + (x^3+2)(1-x^3-x^2-x)$.

        $\gcd(x^4+x+1, x^3 + 2) = 1$ (Multiply the above by $1/3$).

    \subsection[(ii)]{Let $P,Q \in \Z[x]$. Prove that $P$ and $Q$ are relatively prime in $\Q[x]$
        if and only if the ideal $(P,Q)$ of $\Z[x]$ generated by $P$ and $Q$ contains a non-zero integer
        (i.e. $\Z \intersection (P,Q) \neq \set{0})$. Here $(P,Q)$ is the smallest ideal of $\Z[x]$
        containing $P$ and $Q$, $(P,Q) = \set{\alpha P + \beta Q | \alpha, \beta \in \Z[x]}$
    }
        Suppose $P$ and $Q$ are relatively prime in $\Q[x]$.

        By definition of relatively prime, $\lis p(x), q(x) \in \Q[x]$ (henceforth refered to as $p,q$)
        such that $pP + qQ = 1$.

        But $p = a/b$ for some $a \in \Z[x], b \in \N$
        \footnote{Let $b$ be the lowest common multiple of the denominators of the coefficients of
            $p$ in their irreducible fraction forms.}
        ,
        and $q = c/d$ for some $c \in \Z[x], d \in \N$.

        So $pP + qQ = adP + cbQ = bd$, where $ad, cb \in \Z[x], bd \in \N$.
        Therefore $bd \in (P,Q)$.

        Now, Suppose $(P,Q)$ contains a non-zero integer $n \in \Z \setminus\set{0}$.

        So $\lis a,b \in \Z[x]$ such that $aP + bQ = n$.

        So for $a/n, b/n \in \Q[x]$, $a/n P + b/n Q = 1$.

        Therefore $P$ and $Q$ are relatively prime in $\Q[x]$.
        \qed

    \subsection[(iii)]{For which primes $p$ and which integers $n\geq 1$ is the polynomial 
        $x^n - p$ irreducible in $\Q[x]$? Justify your answer.}

        Let $p$ be a prime. Let $n \geq 1$.

        By the Eisenstein Criterion: 
        \footnote{Artin's Algebra, Proposition 12.4.6, Eisenstein Criterion.}
        \\
        Since $p$ divides $-p$, $p^2$ does not divide $-p$, and $p$ does not divide $1$,
        $(1)x^n + (-p) = x^n - p$ in irreducible in $\Q[x]$.
        \qed