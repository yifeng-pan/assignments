\documentclass[10pt, letterpaper, titlepage]{article}

\usepackage{amsmath}
\usepackage{xcolor}

%Header
\usepackage[margin=1in]{geometry}
\usepackage{fancyhdr}
\setlength{\headheight}{22.54448pt}
\pagestyle{fancy}
\lhead{}
\rhead{Yifeng Pan
     \\UCID: 30063828}

%Change lable to letter from number
\renewcommand{\thesubsection}{\alph{subsection}}

%Evaluate for calc
\newcommand*\eval[3]{\left.#1\right\rvert_{#2}^{#3}}

%Absolute Value
\newcommand\abs[1]{\left|#1\right|}

%Title page
\title{STAT 323 Assignment 2}
\author{Instructor: Claudia Marie Mahler
    \\Name: Yifeng Pan
    \\UCID: 30063828}
\date{Summer 2019}

%mean of x
\newcommand\mx{\overline{x}}

%For displaying R code
\usepackage{listings}

\usepackage{amssymb}
\newcommand{\Z}{\mathbb{Z}}

\begin{document}
    \maketitle
    \section{Let $X_1 , X_2 , \hdots , X_{20}$ repensent a random sample where each $X_i$ is a normally distributed with
        a mean $\mu = 100$ and unknown variance $\sigma^2$. From what you know about the distribution 
        of $\frac{(n - 1) s^2}{\sigma^2}$, find an interval that would contain $\sigma^2$ $95\%$ of the time.}
        $\mu = 100, n = 20, df = 19$
        \begin{align*}
            0.95 &= P(a \leq \sigma^2 \leq b)\\
            &= P(1/a \geq 1/\sigma^2 \geq 1/b) \text{ they are all positive.}\\
            &= P((n - 1)s^2/a \geq (n - 1)s^2/\sigma^2 \geq (n - 1)s^2/b)\\
            &= P(19s^2/a \geq \chi^2_{19} \geq 19s^2/b) \\
            &= P(19s^2/b \leq \chi^2_{19} \leq 19s^2/a)\\
            19s^2/b &= qchisq(0.025, 19)\\
            19s^2/a &= qchisq(0.95+0.025, 19) \\
            a &= \frac{19s^2}{qchisq(0.975, 19)} \\
            b &= \frac{19s^2}{qchisq(0.025, 19)}
        \end{align*}
        The $95\%$ interval for $\sigma^2$ would be $(\frac{19s^2}{qchisq(0.975, 19)},\frac{19s^2}{qchisq(0.025, 19)})$.

    \section[]{Let $X_1 , X_2 , \hdots , X_n$ denote a random sample of size $n$ from a population whose density function
        is given by:
        \[
            f(x) = \begin{cases}
                3\beta^3 x^{-4} &\text{for } \beta \leq x\\
                0               &\text{elsewhere}
            \end{cases}
        \]
        where $\beta > 0$ is unknown. Consider the estimator\\
        $\hat \beta = \min(X_1 , X_2 , \hdots , X_n)$. 
        Derive the bias of the estimator $\hat \beta$.
        }
        $F_X(x) = \int_\beta^x{3\beta^3 x^{-4}}dx = (1-\frac{\beta^3}{x^3})$. So 
        $\hat \beta = n[1 - F_X(x)]^{n-1} f_X(x) = n[1-(1-\frac{\beta^3}{x^3})]^{n-1}3\beta^3 x^{-4}
        = n (\frac{\beta^3}{x^3})^{n-1}3\beta^3 x^{-4}
        = 3n \frac{\beta^{3n-3}}{x^{3n-3}} \beta^3 x^{-4}
        = 3n\frac{\beta^{3n}}{x^{3n+1}}$.
        \begin{align*}
            E(\hat \beta) &= E(3n\frac{\beta^{3n}}{x^{3n+1}})\\
            &= \int_\beta^\infty{3n\frac{\beta^{3n}}{x^{3n+1}}x}dx\\
            &= 3 n \beta^{3n} \int_\beta^\infty{\frac{1}{x^{3n}}}dx\\
            &= 3 n \beta^{3n} (-\beta^{-3n+1} / (1-3n)))\\
            &= \beta 3n / (3n - 1)\\
            B(\hat \beta) &= E(\hat \beta) - \beta \\
            &= \beta 3n / (3n - 1) - \beta \\
            &= \beta 3n / (3n - 1) - (3n\beta - \beta)/(3n - 1) \\
            &= \frac{\beta}{3n-1}
        \end{align*}

    \newpage
    \section{Let $X_1 , X_2 , \hdots , X_n$ be a random sample of size $n$ from a population that can be modeled
        by the following probability model:
        \[
            f_X(x) = \frac{\alpha x^{\alpha-1}}{\theta^\alpha}, 0 < x <\theta, \alpha > 0
        \]
        }
        \subsection{Find the probability density function of $X_{(n)} = \max(X_1 , X_2 , \hdots , X_n)$.}
            \begin{align*}
                F_X(x) &= \int_0^x{\frac{\alpha x^{\alpha - 1}}{\theta ^ \alpha}}dx\\
                &= \frac{x^\alpha}{\theta^\alpha}
            \end{align*}
            \begin{align*}
                X_{(n)} &= n(F(x))^{n-1}f(x)\\
                &= n(\frac{x^\alpha}{\theta^\alpha})^{n-1} \frac{\alpha x^{\alpha-1}}{\theta^\alpha}\\
                &= an\frac{x^{\alpha n - 1}}{\theta^{\alpha n}} \text{ for $0 < x < \theta, \alpha > 0, n \in \Z^+$}
            \end{align*}
        \subsection{Is $X_{(n)}$ an unbiased estimator for $\theta$? If not, suggest a function of $X_{(n)}$ that is an 
            unbiased estimator of $\theta$.}
            \begin{align*}
                E(X_{(n)}) &= \int_0^\theta{an\frac{x^{\alpha n}}{\theta^{\alpha n}}}dx\\
                &= \frac{\alpha \theta n}{\alpha n+1}
            \end{align*}
            \begin{align*}
                B(X_{(n)}) &= \frac{\alpha \theta n}{\alpha n+1} - \theta\\
                &= \frac{\alpha \theta n}{\alpha n+1} - \frac{\alpha \theta n + \theta}{\alpha n + 1}\\
                &= \frac{\theta}{\alpha n + 1} \text{ which is biased.}
            \end{align*}
            An unbiased estimator of $\theta$ would be : $\frac{X_{(n)} (\alpha n + 1)}{\alpha n}$. Proof:
            \begin{align*}
                B(\frac{X_{(n)} (\alpha n + 1)}{\alpha n}) 
                &= E(\frac{X_{(n)} (\alpha n + 1)}{\alpha n}) - \theta \\
                &= \frac{\alpha n + 1}{\alpha n}E(X_{(n)}) - \theta \\
                &= \frac{\alpha n + 1}{\alpha n} \frac{\alpha \theta n}{\alpha n+1} - \theta\\
                &= \theta - \theta\\
                &= 0 \text{ which is unbiased.}
            \end{align*}

    \section{Let $X_1 , X_2 , \hdots , X_n$ be a random sample of size $n$ where $X_i \sim Exponential(\beta)$
        and Let $Y = \sum_{i = 1}^n X_i$.}
        \subsection{Show that $Z = \frac{2Y}{\beta}$ is a pivotal quantity.}
            (1) $Z$ is a function of $\beta$. (2) $\beta$ is the only unknown.
            \begin{align*}
                m_Z(t) &= E(\exp(\frac{t2Y}{\beta}))\\
                &= E(\exp(\frac{t2(\sum_{i = 1}^n X_i)}{\beta}))\\
                &= E(\exp(\frac{t2X}{\beta}))^n \text{ as they are indepentent samples}\\
                &= m_X(\frac{2t}{\beta})^n\\
                &= (1-\beta \frac{2t}{\beta})^{(-1)^n}\\
                &= (1-\beta \frac{2t}{\beta})^{-n}\\
                &= (1- 2t)^{-n}
            \end{align*}
            Which is the MGF of a chi-square distro with $df = 2n$. (3) So the distro of $Z$ does not depend on $\beta$.
        \subsection{Use your finds from a) to construct a $95\%$ confidence interval for $\beta$ (hint:
        this will be interms of $Y$ and a chi-square distribution).}
            \begin{align*}
                0.95 &= P(a \leq Z \leq b) \text{ where $a = qchisq(0.025,2n), b = qchisq(.95+0.025,2n)$.}\\
                &= P(a \leq 2Y / \beta \leq b)\\
                &= P(1/a \geq \beta / 2Y \geq 1/b) \text{ As they are all positive.}\\
                &= P(2Y/a \geq \beta \geq 2Y/b)
            \end{align*}
            The $95\%$ confidence interval for $\beta$ would be 
            $(\frac{2Y}{qchisq(.95 + 0.025, 2n)}, \frac{2Y}{qchisq(.025, 2n)})$.

    \newpage
    \section{The file {\color{red} NHLSalaries1314.R} contains data that resulted from a random sample of 
        $n = 70$ professional hockey players who have contracts with NHL teams. Load the data
        into R to answer the following questions.}

        \lstinputlisting[language=R, firstline = 8, numbers = left]{NHLSalaries1314.R}

        \noindent
        Output:
        \begin{verbatim}
            Mean:  2141214 
            SD:  1911294 
            Confidence Inteval for mu: ( 1685482 2596946 )
            Confidence Inteval for sigma: ( 1638777 2293373 )
        \end{verbatim}

        \subsection{Find the sample mean and sample standard deviation of the data.}
            Mean:  2141214 \\
            SD:  1911294

        \subsection{Find a $95\%$ confidence interval for $\mu$, the average salary of an NHL player for the
            2013-2014 season. Interpret the meaning of this interval in the context of the data.}
            $95\%$ confidence Inteval for $\mu$: (1685482, 2596946)\\
            This means that we are $95\%$ confident that the true mean for all NHL players for the
            $2013$ to $2014$ season is between $1685482$ and $2596946$.

        \subsection{Find a $95\%$ confidence interval for $\sigma$, the standard deviation of the distribution of
            NHL player salaries for the 2013-2014 season}
            $95\%$ confidence Inteval for $\sigma$: (1638777, 2293373)

    \newpage
    \section{What is the normal body temperature for healthy humans? A random sample of $130$
        healthy human body temperatures yielded an average temperature of $98.25$ degrees
        and a standard deviation of $0.73$ degrees.}
        \subsection{Find a $99\%$ confidence interval for the average body temperature of healthy people.}
            $n = 130, \mx = 98.25, s = 0.73$\\
            $99\%$ confidence interval:
            \begin{align*}
                \mx \pm t_{a/2, n - 1} \frac{s}{\sqrt{n}}
                &= 98.25 \pm t_{0.005, 129} \frac{0.73}{\sqrt{130}}\\
                &= 98.25 \pm qt(0.005, 129) \frac{0.73}{\sqrt{130}}\\
                &\approx 98.25 \pm 0.1674\\
            \end{align*}
            
        \subsection{Does the interval you obtained in part a) contain the value $98.6$ degrees, the
            accepted average temperature cited by physicians? What conclusions can you
            draw?}
            No, the interval in part a) is too low.\\
            Possible conclusions in order of likelihood:
            \begin{description}
                \item 1. The data collected was wrong, likely didn't account for something.
                \item 2. My calculations in a) are wrong.
                \item 3. Nothing is wrong, and this is an extreme edge case of an `random' sample. 
                    The \\
                    $1 - 2 * pt(-((98.6-98.25) * sqrt(130) / 0.73), 129) \approx 99.99998\%$ 
                    confidence interval would contain $98.6$ degrees. 
                    Altho this is statistical negligeable.
                \item 4. The accepted conventions by physicians are wrong.
            \end{description}

    \newpage
    \section{The Environmental Protection Agency (EPA) has collected data on LC50
        measurements (chemical concentrations that kill $50\%$ of test animals) for certain
        chemicals likely to be found in freshwater rivers and lakes. For certain species of fish,
        the LC50 measurements (in parts per million) for DDT in $12$ experiments were as
        follows: $16, 5, 21, 19, 10, 5, 8, 2, 7, 2, 4, 9$.\\
        Using this data and assuming that LC$50$ measurements have an approximately normal
        distribution, construct a $98\%$ confidence interval for the mean LC50 for DDT.}
        Using the script from q5. Replacing line $1$ with $x = c(16,5,21,19,10,5,8,2,7,2,4,9)$,
        and replacing line $2$ with $conf\_level = .98$ results in the following output:
        \begin{verbatim}
            Mean:  9 
            SD:  6.424385 
            Confidence Inteval for mu: ( 3.959158 14.04084 )
            Confidence Inteval for sigma: ( 4.285091 12.19355 )
        \end{verbatim}
        $98\%$ confidence interval for mean LC50: (3.959158, 14.04084)

    \section{In a recent survey of $n = 1005$ Canadians between the ages of $18$ and $34$, the polling
        company Ipsos found that $723$ indicated that they “owe it to their parents to keep them
        comfortable in their retirement.” Find a $99\%$ confidence interval for the proportion of all
        Canadians aged $18$ to $34$ who hold the same sentiment.}
        $n = 1005, \hat p = 723/1005$, About a normal distro.
        $99\%$ confidence interval:
        \begin{align*}
            \hat p \pm Z_{a/2} \sqrt{\frac{\hat p (1 - \hat p)}{n}}
            &= 723/1005 \pm Z_{0.005} \sqrt{\frac{(723/1005)(1 - 723/1005)}{1005}}\\
            &\approx 0.719403 \pm qnorm(0.005) 0.01417244\\
            &\approx 0.719403 \pm 0.03650578\\
            &\approx (68.29\%, 75.59\%)\\
        \end{align*}


    \newpage
    \section{Owing to the variability of trade-in allowance, the profit per new car sold by an
        automobile dealer varies from car to car. The profits per sale (in hundreds of dollars)
        tabulated for the past week were $2.1, 3.0, 1.2, 6.2, 4.5,$ and $5.1.$ Assume the profits are
        normally distributed.}

        Using the script from q5. Replacing line $1$ with $x = c(2.1,3.0,1.2,6.2,4.5,5.1)$,
        results in the following output:
        \begin{verbatim}
            Mean:  3.683333 
            SD:  1.905168 
            Confidence Inteval for mu: ( 1.683982 5.682685 )
            Confidence Inteval for sigma: ( 1.189221 4.672643 )
        \end{verbatim}

        \subsection{Find a $95\%$ confidence interval for $\mu$, the mean profit of a new car sold by the
            automobile dealer.}
            $95\%$ confidence Inteval for $\mu$: (1.683982, 5.682685)

        \subsection{Find a $95\%$ confidence interval for $\sigma$, the standard deviation of the profit of a new
            car sold by the automobile dealer.}
            $95\%$ confidence Inteval for $\sigma$: (1.189221, 4.672643)

\end{document}
