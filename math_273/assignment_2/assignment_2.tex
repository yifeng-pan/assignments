\documentclass[10pt, letterpaper, titlepage]{article}

\usepackage{amsmath}

%Header
\usepackage[margin=1in]{geometry}
\usepackage{fancyhdr}
\setlength{\headheight}{22.54448pt}
\pagestyle{fancy}
\lhead{}
\rhead{Yifeng Pan
     \\UCID: 30063828}

\usepackage{amssymb}
\newcommand{\0}{\varnothing}

%Change lable to letter from number
\renewcommand{\thesubsection}{\alph{subsection}}

\newcommand{\Z}{\mathbb{Z}}
\newcommand{\R}{\mathbb{R}}
\newcommand{\F}{\mathcal{F}}

\newcommand{\1}{\{ 1 \}}
\newcommand{\2}{\{ (1,1) \}}
\newcommand{\OTO}{one-to-one}
\newcommand{\gof}{g \circ f}
\newcommand{\ZXZ}{\Z ^+ \times \Z^+}

%Title page
\title{MATH 273 Assignment 2}
\author{Instructor: Thi Ngoc Dinh
    \\UCID: 30063828}
\date{Fall 2018}

\usepackage{centernot}
\newcommand{\notmid}{\centernot|}

\begin{document}
    \maketitle

    \section{ }
        \subsection{Use the Euclidean Algorithm to find $gcd(65,18)$ and use that to find integers $x$ and $y$ so that $gcd(65,18) = 65x + 18y$.}
            From Euclidean Algorithm: If $a = qb + r$ where $a, q, b, r \in \Z$. Then $gcd(a,b) = gcd(b,r)$.
            \\
            \begin{tabular}{ c | c | c | c }
                a & q & b & r \\ \hline
                65 & 3 & 18 & 11 \\
                18 & 1 & 11 & 7 \\
                11 & 1 & 7 & 4 \\
                7 & 1 & 4 & 3 \\
                4 & 1 & 3 & 1 \\
                3 & 3 & 1 & 0 
            \end{tabular}
            \\
            $gcd(65,18) = 1$.
            \\
            \begin{tabular}{ c | c | c }
                x & y & 65x+19y\\ \hline
                1 & -3 & 11\\
                -1 & 4 & 7\\
                2 & -7 & 4\\
                -3 & 11 & 3\\
                5 & -18 & 1
            \end{tabular}
            \\
            $gcd(65,18) = 1 = 65(5) \times 18(-18)$.

        \subsection{Is it true that for all integers $a, b,$ and $c$, if $a \mid bc$ then $a \mid b$ or $a \mid c$? Prove your answer.}
            It is true. Proof by contradiction:
            Suppose $a,b,c \in \Z$, such that $a \mid bc$ and $a \notmid b$ and $a \notmid c$.
            This means $b = ad + i$ and $c = ae + j$ where $d,e, i,j \in \Z$ and $a \notmid i$ and $a \notmid j$.
            So $bc = (ad + i)(ae + j) = a^2de + iae + jad + ij = a(ade+ie+jd) + ij$.
            Because $(ade+ie+jd), ij \in \Z$ and $a \notmid ij$,
            $a \notmid bc$ as $bc = a(ade+ie+jd) + ij$.
            Which contradicts the assumption that $a \mid bc$.
            Therefore the statement can not be false.

        \subsection{Is it true that for all integers $x$, if $18 \mid 65x$ then $18 \mid x$? Prove your answer.}
            It is true. Proof: 
            Let $a = 18, b = 65$ and $c = x$ where $c,x \in \Z$, such that $a \mid bc$.
            Because $18 \notmid 65, a \notmid b$.
            From 1.(b): $\forall a,b,c \in \Z$ if $a \mid bc$ then $a \mid b$ or $a \mid c$.
            $a\mid c$ because $a \notmid b$.
            So $18 \mid x$.

        \subsection{Is it true that for all integers $a, b,$ and $c$, if $a \mid bc$ and $gcd(a,b) = 1$ then $a \mid c$? Prove your answer.}
            It is true. Proof:
            Let $a,b,c\in\Z$ so that $a\mid bc$ and $gcd(a,b)=1$.
            Case 1: $a \notmid b$ 
            Then $a \mid c$ because $a \mid bc$ and becasue 1.is true.
            Case 2: $a \mid b$
            If $a \mid b$ then $gcd(a,b) = a$ since $b = da + 0$ where $d \in\Z$, so $gcd(a,b) = gcd(a,0) = a$ for $a \neq 0$.
            Because $a \mid b$ and $gcd(a,b) = 1$, $a = 1$.
            $1 \mid c, \forall c \in\Z$.
            So $a \mid c$.
            Therefore $a \mid c$ in all cases.

    \newpage
    \section{Let $\Z ^+$ be the set of all positive integers and let $R$ be the relation on $\ZXZ$ defined by: For any $(a,b), (c,d) \in \ZXZ, (a,b)R(c,d)$ if and only if $a+2b = c +2d$.}
        \subsection{Prove that $R$ is an equivalence relation on $\ZXZ$.}
            Proof for Reflictive:
            Let $(a,b)\in\ZXZ$.
            $a+2b = a + 2b$, so $(a,b)R(a,b)$.
            \\
            Proof for Symmetric:
            Let $(a,b), (c,d)\in\ZXZ$ such that $(a,b)R(c,d)$.
            $c+2d = a + 2b$ because $a+2b = c + 2d$. So $(c,d)R(a,b)$.
            \\
            Proof for Transitive:
            Let$(a,b),(c,d),(e,f)\in\ZXZ$ such that $(a,b)R(c,d)$ and $(c,d)R(e,f)$.
            $a+2b = c + 2d$ because $(a,b)R(c,d)$.
            $ c + 2d =e +2f$ because $(c,d)R(e,f)$.
            $a+2b = c + 2d =e +2f$, so $(a,b)R(e,f)$.
            $R$ is an equivalence relation because it is reflecctive, symmetric and transitive.

        \subsection{List all elements of $[(3,3)]$, the equivalence class of $(3,3)$.}
            $[(1, 4), (3, 3), (5, 2), (7, 1)]$

        \subsection{Is there an equivalence class that has exactly 13 elements? If there is one, list all elements of that class.}
            Yes.
            $[(1, 13), (3, 12), (5, 11), (7, 10), (9, 9), (11, 8), (13, 7), (15, 6), (17, 5), (19, 4), (21, 3), (23, 2), (25, 1)]$

        \subsection{Is there an equivalence class that has exactly 273 element? Prove your answer.}
            Yes.
            $\forall n \in\Z^+, \exists$ an equivalence class of size $n$ for the relation $R$.
            Proof:
            Let $x \in \Z^+$, choose (1,x).
            The equivalence class of (1,x) has the size of x.
            Proof by example:
            [(1, 273), (3, 272), (5, 271), (7, 270), (9, 269), (11, 268), (13, 267), (15, 266), (17, 265), (19, 264), (21, 263), (23, 262), (25, 261), (27, 260), (29, 259), (31, 258), (33, 257), (35, 256), (37, 255), (39, 254), (41, 253), (43, 252), (45, 251), (47, 250), (49, 249), (51, 248), (53, 247), (55, 246), (57, 245), (59, 244), (61, 243), (63, 242), (65, 241), (67, 240), (69, 239), (71, 238), (73, 237), (75, 236), (77, 235), (79, 234), (81, 233), (83, 232), (85, 231), (87, 230), (89, 229), (91, 228), (93, 227), (95, 226), (97, 225), (99, 224), (101, 223), (103, 222), (105, 221), (107, 220), (109, 219), (111, 218), (113, 217), (115, 216), (117, 215), (119, 214), (121, 213), (123, 212), (125, 211), (127, 210), (129, 209), (131, 208), (133, 207), (135, 206), (137, 205), (139, 204), (141, 203), (143, 202), (145, 201), (147, 200), (149, 199), (151, 198), (153, 197), (155, 196), (157, 195), (159, 194), (161, 193), (163, 192), (165, 191), (167, 190), (169, 189), (171, 188), (173, 187), (175, 186), (177, 185), (179, 184), (181, 183), (183, 182), (185, 181), (187, 180), (189, 179), (191, 178), (193, 177), (195, 176), (197, 175), (199, 174), (201, 173), (203, 172), (205, 171), (207, 170), (209, 169), (211, 168), (213, 167), (215, 166), (217, 165), (219, 164), (221, 163), (223, 162), (225, 161), (227, 160), (229, 159), (231, 158), (233, 157), (235, 156), (237, 155), (239, 154), (241, 153), (243, 152), (245, 151), (247, 150), (249, 149), (251, 148), (253, 147), (255, 146), (257, 145), (259, 144), (261, 143), (263, 142), (265, 141), (267, 140), (269, 139), (271, 138), (273, 137), (275, 136), (277, 135), (279, 134), (281, 133), (283, 132), (285, 131), (287, 130), (289, 129), (291, 128), (293, 127), (295, 126), (297, 125), (299, 124), (301, 123), (303, 122), (305, 121), (307, 120), (309, 119), (311, 118), (313, 117), (315, 116), (317, 115), (319, 114), (321, 113), (323, 112), (325, 111), (327, 110), (329, 109), (331, 108), (333, 107), (335, 106), (337, 105), (339, 104), (341, 103), (343, 102), (345, 101), (347, 100), (349, 99), (351, 98), (353, 97), (355, 96), (357, 95), (359, 94), (361, 93), (363, 92), (365, 91), (367, 90), (369, 89), (371, 88), (373, 87), (375, 86), (377, 85), (379, 84), (381, 83), (383, 82), (385, 81), (387, 80), (389, 79), (391, 78), (393, 77), (395, 76), (397, 75), (399, 74), (401, 73), (403, 72), (405, 71), (407, 70), (409, 69), (411, 68), (413, 67), (415, 66), (417, 65), (419, 64), (421, 63), (423, 62), (425, 61), (427, 60), (429, 59), (431, 58), (433, 57), (435, 56), (437, 55), (439, 54), (441, 53), (443, 52), (445, 51), (447, 50), (449, 49), (451, 48), (453, 47), (455, 46), (457, 45), (459, 44), (461, 43), (463, 42), (465, 41), (467, 40), (469, 39), (471, 38), (473, 37), (475, 36), (477, 35), (479, 34), (481, 33), (483, 32), (485, 31), (487, 30), (489, 29), (491, 28), (493, 27), (495, 26), (497, 25), (499, 24), (501, 23), (503, 22), (505, 21), (507, 20), (509, 19), (511, 18), (513, 17), (515, 16), (517, 15), (519, 14), (521, 13), (523, 12), (525, 11), (527, 10), (529, 9), (531, 8), (533, 7), (535, 6), (537, 5), (539, 4), (541, 3), (543, 2), (545, 1)]

    \newpage
    \section{Let $A = \{ 1,2,3,4 \}$. Let $\F$ be the set of all functions from $A$ to $A$. Let $R$ be the relation on $\F$ defined by: For any $f,g \in \F, fRg$ if and only if $f(i) = g(i)$ for some $i \in A$.}
        \subsection{Is $R$ reflictive? symmetric? transitive? Prove your answer.}
            It is reflictive. Proof:
            Let $f \in \F$.
            Let $i\in A$.
            $f(i) = f(i)$.
            \\
            It is symmetric. Proof:
            Let $f,g \in \F$ such that $fRg$.
            Let $i\in A$.
            $g(i) = f(i)$ because $f(i) =g(i)$, so $gRf$.
            \\
            It is transitive. Proof:
            Let $f,g,h\in\F$ such that $fRg, gRh$.
            Let $i\in A$.
            $f(i) = g(i)$ because $fRg$.
            $g(i) = h(i)$ because $gRh$.
            $f(i) = g(i) = h(i)$, so $fRh$

        \subsection{Is it true that for all functions $f \in \F$, there exists a function $g \in \F$ so that $fRg$? Prove your answer.}
            Yes. Proof:
            Let $f\in\F$.
            Choose $g = f$.
            Becasue $R$ is reflective from 3.(a), $fRg$.

        \subsection{Is it true that there exists a function $g \in \F$ so that for all functions $f \in \F, fRg$? Prove your answer.}
            No. Proof by contradiction:
            Suppose $\exists g \in\F$ so that $\forall f\in\F, fRg$. 
            Because $R$ is reflective, symmetric, and transitive from 3.(a), it is an equvalence relation.
            Because $\forall f\in\F, fRg$ and $R$ is an equivalence relation, $R$ has only one equivalence class.
            Let $i \in A$.
            Let $h,p \in \F$, defined as: $h(i) = 1, p(i) = 2$.
            $h(i) = 1 \neq 2 = g(i)$, therefore $(h,g) \not\in R$.
            This means $R$ has more then one equivalence class. 
            This contradicts the fact that it has only one equivalence class.
            Therefore the assumption cannot be true, and there exists no such $g$. 

\end{document}