\documentclass[10pt, letterpaper, titlepage]{article}

%Size of section header
\usepackage{sectsty}
\sectionfont{\fontsize{12}{15}\selectfont}

%quattrocento font
\usepackage[sfdefault]{quattrocento}

\usepackage[nosfdefault]{comicneue}

\usepackage{amsmath}
\usepackage{amssymb}
\usepackage{xcolor}
\usepackage{multicol}

%Fitch-style proofs
\usepackage{lplfitch}
\setlength{\fitchprfwidth}{5.6cm}
\setlength{\fitchctxwidth}{6.1cm}
\newcommand*\lx[1]{{\bf X:} #1}

%DEBUG
%Getting \hbox too wide errors.
%check https://en.wikibooks.org/wiki/LaTeX/Errors_and_Warnings
%\overfullrule=.01cm
%Problem with the lplfitch package
%It doesnt like nested proofs

% For all metavariables
\usepackage[cal=boondoxo]{mathalfa}
\newcommand*{\meta}[1]{\ensuremath{\mathcal{#1}}}

%Header
\usepackage[margin=0.75in]{geometry}
\usepackage{fancyhdr}
\setlength{\headheight}{22.54448pt}
\pagestyle{fancy}
\lhead{}
\rhead{UCID: 30063828}

%Change lable to letter from number
\renewcommand{\thesubsection}{\alph{subsection}}

%Title page
\title{PHIL 377 Assignment 3}
\author{Instructor: Gillman Glyn Payette
    \\UCID: 30063828}
\date{Summer 2019}

\begin{document}
    \maketitle
    
    %P1
    \section{For each item below, provide a Fitch-style proof of the claim.}
        \subsection{$\vdash \lall x A(x)\liff \lall x(A(x)\lor A(x))$}
            \begin{center}
                \fitchctx{
                    \subproof{
                        \pline[1. ]{\lall x A(x)}
                    }{
                        \pline[2. ]{A(c)}[\lalle{1}] \\
                        \pline[3. ]{A(c) \lor A(c)}[\lori{2}] \\
                        \pline[4. ]{\lall x (A(x) \lor A(x))}[\lalli{3}]
                    }
        
                    \subproof{
                        \pline[5. ]{\lall x (A(x) \lor A(x))}
                    }{
                        \pline[6. ]{A(c) \lor A(c)}[\lalle{5}] \\
                        \subproof{
                            \pline[7. ]{A(c)}
                        }{
                            \pline[8. ]{A(c) \land A(c)}[\landi{7}, 7]\\
                            \pline[9. ]{A(c)}[\lande{8}]
                        }
                        \pline[10. ]{A(c)}[\lore{6}{7--9}{7--9}]\\
                        \pline[11. ]{\lall x A(x)}[\lalli{10}]
                    }
                    \pline[12. ]{\lall x A(x) \liff \lall x (A(x) \lor A(x))}[\liffi{1--4}{5--11}]
                }
            \end{center}
    
        \subsection{$\vdash \lall x\lall y (x=y \lif (G(x,y)\liff G(y,x)))$}
            \begin{center}
                \fitchctx{
                    \subproof{
                        \pline[1. ]{i = j}
                    }{
                        \subproof{
                            \pline[2. ]{G(i,j)}
                        }{
                            \pline[3. ]{G(i,i)}[\eqe{1}{2}]\\
                            \pline[4. ]{G(j,i)}[\eqe{1}{3}]
                        }
                        \subproof{
                            \pline[5. ]{G(j,i)}
                        }{
                            \pline[6. ]{G(j,j)}[\eqe{1}{5}]\\
                            \pline[7. ]{G(i,j)}[\eqe{1}{6}]
                        }
                        \pline[8. ]{G(i,j) \liff G(j,i)}[\liffi{2--4}{5--7}]
                    }
                    \pline[9. ]{i = j \lif (G(i,j) \liff G(j,i))}[\lifi{1--8}]\\
                    \pline[10. ]{\lall y(i = y \lif (G(i,y) \liff G(y,i)))}[\lalli{9}]\\
                    \pline[11. ]{\lall x \lall y(x = y \lif (G(x,y) \liff G(y,x)))}[\lalli{10}]
                }
            \end{center}
        
        \begin{multicols}{2}
            \subsection{$\lall x (F(x)\lor G(x))$\\
                $\vdash \lis x \neg F(x)\lif \lis x G(x)$}
                \fitchprf{
                    \pline[1. ]{\lall x (F(x)\lor G(x))}
                }{
                    \subproof{
                        \pline[2. ]{\lis x \lnot F(x)}
                    }{
                        \subproof{
                            \pline[3. ]{\lnot F(c)}
                        }{
                            \pline[4. ]{F(c) \lor G(c)}[\lalle{1}]\\
                            \subproof{
                                \pline[5. ]{F(c)}
                            }{
                                \pline[6. ]{\lfalse}[\lnote{3,5}]\\
                                \pline[7. ]{G(c)}[\lx{6}]
                            }
                            \subproof{
                                \pline[8. ]{G(c)}
                            }{
                                \pline[9. ]{G(c) \land G(c)}[\landi{8}, 8]\\
                                \pline[10. ]{G(c)}[\lande{9}]
                            }
                            \pline[11. ]{G(c)}[\lore{4}{5--7}{8--10}]\\
                            \pline[12. ]{\lis x G(x)}[\lexii{11}]
                        }
                        \pline[13. ]{\lis x G(x)}[\lexie{3}{4--12}]
                    }
                    \pline[14. ]{\lis x \lnot F(x) \lif \lis x G(x)}[\lifi{2--13}]
                }
                
            \subsection{$\lall z[G(z)\lif \lall y (K(y)\lif H(z,y))], $\\
                $(K(i)\land G(j))\land i=j$\\
                $\vdash H(i,i)$}
                \fitchprf{
                    \pline[1. ]{\lall z[G(z)\lif \lall y (K(y)\lif H(z,y))]}\\
                    \pline[2. ]{(K(i)\land G(j))\land i=j}
                }{
                    \pline[3. ]{G(j)\lif \lall y (K(y)\lif H(j,y))}[\lalle{1}]\\
                    \pline[4. ]{K(i) \land G(j)}[\lande{2}]\\
                    \pline[5. ]{G(j)}[\lande{4}]\\
                    \pline[6. ]{\lall y (K(y)\lif H(j,y))}[\life{3}{5}]\\
                    \pline[7. ]{K(i)\lif H(j,i)}[\lalle{6}]\\
                    \pline[8. ]{K(i)}[\lande{4}]\\
                    \pline[9. ]{H(j,i)}[\life{7}{8}]\\
                    \pline[10. ]{i = j}[\lande{2}]\\
                    \pline[11. ]{H(i,i)}{\eqe{9}{10}}
                }
        \end{multicols}

    \newpage
    %P2
    \section[]{For each of the following claims, 
        \begin{itemize}
            \item[i)] If it is true, provide a Fitch-style proof of the claim,
            \item[ii)] If it is not true, provide an interpretation that is a counterexample to the claim.
        \end{itemize}
    }
        \begin{multicols}{2}
            \subsection{$\lis x F(x,a), $\\
                $\lall y (y=a\lif y=b)$\\
                $\vdash \lis y F(y,y)$}
                False. Proof by example:\\
                Let the domain comprise of the letters $a, b$ and $c$.\\
                Suppose $F(c,a)$ and $a = b$ are true.\\
                Then $\lis x F(x,a)$ is true.\\
                And $\lall y (y=a\lif y=b)$ is also true.\\
                However $\lis y F(y,y)$ is false.\\
                Therefore $\lis x F(x,a)$ and $\lall y (y=a\lif y=b)$ 
                \\cannot prove $\lis y F(y,y)$.
    
            \subsection{$L(a)\liff \lall x L(x)$\\
                $\dashv\vdash \lis x L(x)$}
                False. Proof by example:\\
                Let the domain comprise of the letters $a$ and $b$.\\
                Suppose $L(a)$ and $\lnot L(b)$ are true.\\
                Then $\lis x L(x)$ is true.\\
                But, $L(a)\liff \lall x L(x)$ is false,\\
                because $L(a)$ is true while $\lall x L(x)$ is false.\\
                Therefore $\lis x L(x)$ cannot prove $L(a)\liff \lall x L(x)$.\\
                Therefore they are not provably equivalent.
    
            \subsection{$\lall x (F(x)\lif \lis y(G(y,x) \land \neg G(x,y))),$\\
                $\lis x F(x)$\\
                $\vdash \lis x\lis y \neg x=y$}
                \begin{center}
                    True.\\
                    \fitchprf{
                        \pline[1. ]{\lall x (F(x)\lif \lis y(G(y,x) \land \neg G(x,y)))}\\
                        \pline[2. ]{\lis x F(x)}
                    }{
                        \subproof{
                            \pline[3. ]{F(c)}
                        }{
                            \pline[4. ]{F(c)\lif \lis y(G(y,c) \land \neg G(c,y))}[\lalle{1}]\\
                            \pline[5. ]{\lis y(G(y,c) \land \neg G(c,y))}[\life{4}{3}]\\
                            \subproof{
                                \pline[6. ]{G(d,c) \land \neg G(c,d)}
                            }{
                                \subproof{
                                    \pline[7. ]{c = d}
                                }{
                                    \pline[8. ]{G(d,c)}[\lande{6}]\\
                                    \pline[9. ]{G(c,c)}[\eqe{7}{8}]\\
                                    \pline[10. ]{\lnot G(c,d)}[\lande{6}]\\
                                    \pline[11. ]{\lnot G(c,c)}[\eqe{7}{10}]\\
                                    \pline[12. ]{\lfalse}[\lnote{9}, {11}]
                                }
                                \pline[13. ]{\lnot c=d}[\lnoti{7--12}]\\
                                \pline[14. ]{\lis y \lnot c = y}[\lexii{13}]\\
                                \pline[15. ]{\lis x \lis y \lnot x = y}[\lexii{14}]
                            }
                            \pline[16. ]{\lis x \lis y \lnot x = y}[\lexie{5}{6--15}]
                        }
                        \pline[17. ]{\lis x \lis y \lnot x = y}[\lexie{5}{3--16}]
                    }
                \end{center}
    
            \subsection{$\lall x F(x,a)\liff \neg \lall x \neg G(x,b),$\\
                $\lis x(F(x,a)\land \neg G(x,b))$\\
                $\vdash \bot$}
                False. Proof by example:\\
                Let the domain comprise of the letters $i, j, a$ and $b$.\\
                Suppose $F(i,a), \lnot F(j, a)$ and $\lnot G(i, b)$ are true. \\
                Then $\lall x F(x,a)\liff \neg \lall x \neg G(x,b)$ is vacuously true, \\
                as both $\lall x F(x,a)$ and $\neg \lall x \neg G(x,b)$ are false.\\
                Now, $\lis x(F(x,a)\land \neg G(x,b))$ is also true.\\
                Therefore $\lall x F(x,a)\liff \neg \lall x \neg G(x,b)$ and \\
                    $\lis x(F(x,a)\land \neg G(x,b))$ cannot prove a contradiction.
    
            \subsection{$\lall y (H(y)\land (J(y,y)\land M(y)))$\\
                $\vdash \lis x J(x,b)\land \lall x M(x)$}
                True.\\
                \fitchprf{
                    \pline[1. ]{\lall y [H(y) \land (J(y,y) \land M(y))]}
                }{
                    \pline[2. ]{H(b) \land (J(b,b) \land M(b))}[\lalle{1}]\\
                    \pline[3. ]{J(b,b) \land M(b)}[\lande{2}]\\
                    \pline[4. ]{J(b,b)}[\lande{3}]\\
                    \pline[5. ]{\lis x J(x,b)}[\lexii{4}]\\
                    \pline[6. ]{M(b)}[\lande{3}]\\
                    \pline[7. ]{\lall x M(x)}[\lalli{6}]\\
                    \pline[8. ]{\lis x J(x,b)\land \lall x M(x)}[\landi{5}, 7]
                }
        \end{multicols}

    \newpage
    %P3
    \section[]{Find an example of an invalid argument where only 
        $\lall$I  is used improperly because the name that is quantified away still remains in the formula, i.e., 
        \begin{center}
            \fitchctx{
                \pline[m. ]{\meta{A (\ldots a \ldots a \ldots)}} \\
                \pline[n. ]{\lall x\meta{A (\ldots x \ldots a \ldots)}} [\lalli{m}]
            }
        \end{center}
    }
        Invalid argument in the form of a invalid Fitch-style proof.
        \begin{center}
            \fitchprf{
                \pline[1. ]{\lall x F(x, x)}
            }{
                \pline[2. ]{F(a, a)}[\lalle 1]\\
                \pline[3. ]{\lall x F(x, a)}[\lalli 2 {\color{red}(bad)}]
            }
        \end{center}
        Therefore $\lall x F(x, x) \vdash \lall x F(x, a)$.\\
        However this is an invalid argument. Proper proof by example:\\
        Let the domain comprise of the letters $a$ and $b$.\\
        Suppose $F(a, a), F(b, b)$ and $\lnot F(b, a)$ are true.\\
        Then $\lall x F(x, x)$ is true.\\
        However $\lall x F(x, a)$ is false.\\
        Therefore $\lall x F(x, x)$ cannot prove $\lall x F(x, a)$.\\
        Therefore it's possible for the premises to be true and the conclusion to be false.\\
        Therefore it's an invalid argument.

    %P4
    \section{Bonus: Provide an informal proof of the following claim:
        \[\lall x\lall y (R(x,y)\lif R(y,x)), \lall x\lall y\lall z((R(x,y)\land R(y,z))\lif R(x,z)), 
        \lall x \lis y R(x,y)\vDash \lall x R(x,x) \]}

        \subsection*{\comicneue\textbf{Mathematical proof where $R$ is a relation over the set $S$:}}
        {\comicneue
            $\lall x\lall y (R(x,y)\lif R(y,x))$ 
            means $\lall x, y \in S$: if $xRy$ then $yRx$. 
            Therefore $R$ is symmetric.\\
            $\lall x\lall y\lall z((R(x,y)\land R(y,z))\lif R(x,z))$
            means $\lall x, y, z \in S$: if $xRy$ and $yRz$ then $xRz$.
            Therefore $R$ is transitive.\\ 
            And lastly, we know $\lall x \lis y R(x,y)$,
            this means $\lall x \in S, \lis y \in S$, such that $xRy$.\\
            Now let $a$ be an arbitrary element of $S$. 
            Then we know $\exists b$ such that $aRb$.\\
            Because $aRb$ and $R$ is symmetric: $bRa$.\\
            Because $aRb$ and $bRa$ and $R$ is transitive: $aRa$.\\
            Therefore $\lall a \in S$: $aRa$. 
            \\Therefore $R$ is reflexive, or $\lall x R(x,x)$.
        }

        %The way this is done is very bad.
        %So many warnings
        \subsection*{\comicneue\textbf{Fitch-style Proof: }}
            \begin{center}
                \comicneue
                \fitchprf{
                    \pline[1. ]{\lall x\lall y (R(x,y)\lif R(y,x))}\\
                    \pline[2. ]{\lall x\lall y\lall z((R(x,y)\land R(y,z))\lif R(x,z))}\\
                    \pline[3. ]{\lall x \lis y R(x,y)}
                }{
                    \pline[4. ]{\lall y (R(a,y)\lif R(y,a))}[\textbf{\comicneue $\lall$ Elim: }1]\\
                    \pline[5. ]{R(a,b)\lif R(b,a)}[\textbf{\comicneue $\lall$ Elim: }4]\\
                    \pline[6. ]{\lall y\lall z((R(a,y)\land R(y,z))\lif R(a,z))}[\textbf{\comicneue $\lall$ Elim: }2]\\
                    \pline[7. ]{\lall z((R(a,b)\land R(b,z))\lif R(a,z))}[\textbf{\comicneue $\lall$ Elim: }6]\\
                    \pline[8. ]{(R(a,b)\land R(b,a))\lif R(a,a)}[\textbf{\comicneue $\lall$ Elim: }7]\\
                    \pline[9. ]{\lis y R(a,y)}[\textbf{\comicneue $\lall$ Elim: }3]\\
                    \subproof{
                        \pline[10. ]{R(a,b)}
                    }{
                        \pline[11. ]{R(b,a)}[\textbf{\comicneue $\lif$ Elim: }5, 10]\\
                        \pline[12. ]{R(a,b) \land R(b,a)}[\textbf{\comicneue $\land$ Intro: }10, 11]\\
                        \pline[13. ]{R(a,a)}[\textbf{\comicneue $\lif$ Elim: }8, 12]
                    }
                    \pline[14. ]{R(a,a)}[\textbf{\comicneue $\lis$ Elim: }9, 10--13]\\
                    \pline[15. ]{\lall x R(x,x)}[\textbf{\comicneue $\lall$ Intro: }14]
                }
            \end{center}
\end{document}